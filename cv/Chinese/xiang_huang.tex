%%%%%%%%%%%%%%%%%%%%%%%%%%%%%%%%%%%%%%%%%%%%%%%%%%%%%%%%%%%%%%%%%%%%%%%%
%%%%%%%%%%%%%%%%%%%%%% Chinese CV Translation %%%%%%%%%%%%%%%%%%%%%%%%
%%%%%%%%%%%%%%%%%%%%%%%%%%%%%%%%%%%%%%%%%%%%%%%%%%%%%%%%%%%%%%%%%%%%%%%%

%%%%%%%%%%%%%%%%%%%%%%%%%%%% Document Setup %%%%%%%%%%%%%%%%%%%%%%%%%%%%

% Don't like 10pt? Try 11pt or 12pt
\documentclass[10pt]{article}
% !TeX program  = XeLaTeX

% This is a helpful package that puts math inside length specifications
\usepackage{calc}
\usepackage{comment}
\usepackage{setspace}
\usepackage{graphicx}
\usepackage[UTF8]{ctex} % for Chinese support
\usepackage{etaremune} % reverse enumberate
\graphicspath{ {images/}} %upload your signature to this file

% Simpler bibsection for CV sections
% (thanks to natbib for inspiration)
\makeatletter
\newlength{\bibhang}
\setlength{\bibhang}{1em} %1em}
\newlength{\bibsep}
 {\@listi \global\bibsep\itemsep \global\advance\bibsep by\parsep}
\newenvironment{bibsection}%
        {\begin{enumerate}{}{%
%        {\begin{list}{}{%
       \setlength{\leftmargin}{\bibhang}%
       \setlength{\itemindent}{-\leftmargin}%
       \setlength{\itemsep}{\bibsep}%
       \setlength{\parsep}{\z@}%
        \setlength{\partopsep}{0pt}%
        \setlength{\topsep}{0pt}}}
        {\end{enumerate}\vspace{-.6\baselineskip}}
%        {\end{list}\vspace{-.6\baselineskip}}
\makeatother

% Layout: Puts the section titles on left side of page
\reversemarginpar

%% Use these lines for letter-sized paper
\usepackage[paper=letterpaper,
            %includefoot, % Uncomment to put page number above margin
            marginparwidth=1.2in,     % Length of section titles
            marginparsep=.05in,       % Space between titles and text
            margin=1in,               % 1 inch margins
            includemp]{geometry}

%% More layout: Get rid of indenting throughout entire document
\setlength{\parindent}{0in}

\usepackage[shortlabels]{enumitem}
\usepackage{lastpage} % for the page number

\usepackage{fancyhdr,lastpage}
\pagestyle{fancy}
%\pagestyle{empty}      % Uncomment this to get rid of page numbers
\fancyhf{}\renewcommand{\headrulewidth}{0pt}
\fancyfootoffset{\marginparsep+\marginparwidth}
\newlength{\footpageshift}
\setlength{\footpageshift}
          {0.5\textwidth+0.5\marginparsep+0.5\marginparwidth-2in}
\lfoot{\hspace{\footpageshift}%
       \parbox{4in}{\, \hfill %
               %     \arabic{page} of \protect\pageref*{LastPage} % +LP
%                    \arabic{page}                               % -LP
                    \hfill \,}}

% Finally, give us PDF bookmarks
\usepackage{color,hyperref}
\definecolor{darkblue}{rgb}{0.0,0.0,0.3}
\hypersetup{colorlinks,breaklinks,
            linkcolor=darkblue,urlcolor=darkblue,
            anchorcolor=darkblue,citecolor=darkblue}

%%%%%%%%%%%%%%%%%%%%%%%% End Document Setup %%%%%%%%%%%%%%%%%%%%%%%%%%%%


%%%%%%%%%%%%%%%%%%%%%%%%%%% Helper Commands %%%%%%%%%%%%%%%%%%%%%%%%%%%%

% The title (name) with a horizontal rule under it
\newcommand{\makeheading}[2][]%
        {\hspace*{-\marginparsep minus \marginparwidth}%
         \begin{minipage}[t]{\textwidth+\marginparwidth+\marginparsep}%
             {\large \bfseries #2 \hfill #1}\\[-0.15\baselineskip]%
                 \rule{\columnwidth}{1pt}%
         \end{minipage}}

% The section headings
\renewcommand{\section}[1]{\pagebreak[3]%
    \hyphenpenalty=10000%
    \vspace{1.3\baselineskip}%
    \phantomsection\addcontentsline{toc}{section}{#1}%
    \noindent\llap{\scshape\smash{\parbox[t]{\marginparwidth}{\raggedright #1}}}%
    \vspace{-\baselineskip}\par}

% An itemize-style list with lots of space between items
\newenvironment{outerlist}[1][\enskip\textbullet]%
        {\begin{itemize}[#1,leftmargin=*]}{\end{itemize}%
         \vspace{-.6\baselineskip}}

% An environment IDENTICAL to outerlist that has better pre-list spacing
% when used as the first thing in a \section
\newenvironment{lonelist}[1][\enskip\textbullet]%
        {\begin{list}{#1}{%
        \setlength{\partopsep}{0pt}%
        \setlength{\topsep}{0pt}}}
        {\end{list}\vspace{-.6\baselineskip}}

% An itemize-style list with little space between items
\newenvironment{innerlist}[1][\enskip\textbullet]%
        {\begin{itemize}[#1,leftmargin=*,parsep=0pt,itemsep=0pt,topsep=0pt,partopsep=0pt]}
        {\end{itemize}}

% An environment IDENTICAL to innerlist that has better pre-list spacing
% when used as the first thing in a \section
\newenvironment{loneinnerlist}[1][\enskip\textbullet]%
        {\begin{itemize}[#1,leftmargin=*,parsep=0pt,itemsep=0pt,topsep=0pt,partopsep=0pt]}
        {\end{itemize}\vspace{-.6\baselineskip}}

% To add some paragraph space between lines.
\newcommand{\blankline}{\quad\pagebreak[3]}
\newcommand{\halfblankline}{\quad\vspace{-0.5\baselineskip}\pagebreak[3]}

% Uses hyperref to link DOI
\newcommand\doilink[1]{\href{http://dx.doi.org/#1}{#1}}
\newcommand\doi[1]{doi:\doilink{#1}}

% For \url{SOME_URL}, links SOME_URL to the url SOME_URL
\providecommand*\url[1]{\href{#1}{#1}}
% Same as above, but pretty-prints SOME_URL in teletype fixed-width font
\renewcommand*\url[1]{\href{#1}{\texttt{#1}}}

% For \email{ADDRESS}, links ADDRESS to the url mailto:ADDRESS
\providecommand*\email[1]{\href{mailto:#1}{#1}}

\providecommand\BibTeX{{B\kern-.05em{\sc i\kern-.025em b}\kern-.08em
    \TeX}}
\providecommand\Matlab{\textsc{Matlab}}

%Skype information
\newcommand*{\Skype}{\href{skype:xiangyazi24?add}{xiangyazi24}}
\newcommand{\Absender}[1][\normalsize]{\Skype}

%%%%%%%%%%%%%%%%%%%%%%%% End Helper Commands %%%%%%%%%%%%%%%%%%%%%%%%%%%

\begin{document}
\makeheading{\Large{黄湘} \hfill 伊利诺伊大学斯普林菲尔德分校}

\section{联系方式}

\begin{center}
\begin{tabular}{l l}
 计算机科学系 & \href{mailto:xhuan5@uis.edu}{xhuan5@uis.edu} \\
 3115 UHB, One University Plaza & 电话: +1 (217) 206-8336 \\
 Springfield, IL 62703-5407, USA &
\end{tabular}
\end{center}

\section{当前职位}
助理教授,伊利诺伊大学斯普林菲尔德分校,美国伊利诺伊州斯普林菲尔德(2020年8月至今)。

\section{个人网站}
\href{https://www.xianghuang.org}{\textbf{xianghuang.org}}

\section{研究兴趣}
算法信息论、模拟计算、DNA/分子编程、正规数与理论计算机科学。

\section{访问职位}
访问学者,加州理工学院(2024年8月至2024年12月,\href{https://www.dna.caltech.edu/~winfree/}{Erik Winfree}实验室)。 \\
访问助理教授,勒莫因学院,纽约州雪城市(2019年9月至2020年6月)。

\section{教育背景}

\href{http://www.iastate.edu/}{\textbf{爱荷华州立大学}},美国爱荷华州
\begin{outerlist}
\item[] {计算机科学博士,2020年。}
        \begin{innerlist}
        \item 论文:\emph{Chemical Reaction Networks: Computability, Complexity, and Randomness}。
        \item 导师:\href{http://jacklutz.com}{Jack H. Lutz}。
        \end{innerlist}
\end{outerlist}

\href{http://english.is.cas.cn/}{\textbf{中国科学院软件研究所}}
\begin{outerlist}
\item[] {计算机科学,2009年9月至2012年6月。}
        \begin{innerlist}
        \item 研究方向:\emph{模型检验、形式化方法、自动机理论}。
        \end{innerlist}
\end{outerlist}

\href{http://www.nju.edu.cn}{\textbf{南京大学}}
\begin{outerlist}
\item[]{软件工程学士,2005年9月至2009年6月}。
\end{outerlist}

\section{资助与基金}
\vspace{5mm}
外部资助:
\begin{etaremune}
   \item 项目负责人(PI):\textit{Towards A Hierarchy of Real Numbers Computable by CRN},\$400,000,美国能源部EXPRES项目基金,2023--2025。
\end{etaremune}

伊利诺伊大学斯普林菲尔德分校及伊利诺伊大学系统内部资助:
\begin{etaremune}
    \item 国立台湾大学-伊利诺伊大学系统访问基金项目,\$5,000,2024年。
    \item 竞争性学术研究基金,\$5,000,2023--2024。
    \item 基金申请指导奖,\$1,500,2022--2023。
    \item Leadership Lived Experience (LLE)学生就业计划,\$4,000,2022年。
\end{etaremune}

\section{期刊论文}
\begin{etaremune}
    \item Xiang Huang, Jack H. Lutz, Elvira Mayordomo, and Donald M. Stull, ``Asymptotic divergences and strong dichotomy,'' \textit{IEEE Transactions on Information Theory} 67 (2021), pp. 6296--6305.
    \item Xiang Huang, Titus H. Klinge, James I. Lathrop, Xiaoyuan Li and Jack H. Lutz, ``Real-Time Computability of Real Numbers by Chemical Reaction Networks,'' \textit{Natural Computing} 18(1) (2019), pp. 63--73 \textbf{(invited paper)}.
\end{etaremune}

\section{会议论文}
\begin{etaremune}
  \item[] (带下划线的为指导的学生)
  \item Xiang Huang and \underline{Rachel Huls}, ``Computing Real Numbers with Large-Population Protocols Having a Continuum of Equilibria,'' \textit{The 28th International Conference on DNA Computing and Molecular Programming} (DNA 28, Albuquerque, NM, August 8--12, 2022).
  \item Xiang Huang, Jack H. Lutz, Elvira Mayordomo, and Donald M. Stull, ``Asymptotic divergences and strong dichotomy,'' \textit{Proceedings of the Thirty-seventh Symposium on Theoretical Aspects of Computer Science} (STACS 2020, Montpellier, France, March 10--13, 2020).
  \item Xiang Huang, Jack H. Lutz, and Andrei N. Migunov, ``Algorithmic Randomness in Continuous-Time Markov Chains,'' \textit{Proceedings of the 57th Annual Allerton Conference on Communication, Control, and Computing} (2019).
  \item Xiang Huang, Titus H. Klinge, and James I. Lathrop, ``Real-Time Equivalence of Chemical Reaction Networks and Analog Computers,'' \textit{DNA Computing and Molecular Programming} (DNA 2019), Lecture Notes in Computer Science, vol. 11648, Springer, Cham.
  \item Xiang Huang, Titus H. Klinge, James I. Lathrop, Xiaoyuan Li, and Jack H. Lutz, ``Real-Time Computability of Real Numbers by Chemical Reaction Networks,'' \textit{Proceedings of the 16th International Conference on Unconventional Computation and Natural Computation} (UCNC 2017), pp. 29--40.
  \item Xiang Huang and Donald M. Stull, ``Polynomial Space Randomness in Analysis,'' \textit{Proceedings of the 41st International Symposium on Mathematical Foundations of Computer Science} (MFCS 2016), 86:1--86:13.
\end{etaremune}

\section{同行评审的研讨会论文/扩展摘要}
\begin{etaremune}
  \item Xiang Huang and Andrei N. Migunov, ``A General Purpose Analog Computer to Population Protocol Compiler,'' \textit{In Proceedings of the 21st ACM International Conference on Computing Frontiers Workshops and Special Sessions (CF '24 Companion)}, May 2024.
\end{etaremune}

\section{教材章节}
\begin{etaremune}
    \item Xiang Huang, ``Deterministic Chemical Reaction Network,'' completed chapter for \textit{The Art of Molecular Programming}. Part of a DNA/molecular programming community initiative to create a comprehensive molecular programming textbook (\href{https://molecularprogrammers.org/#aomp}{molecularprogrammers.org}).
\end{etaremune}

\section{奖项}
\begin{etaremune}
  \item 纳米科学、计算与工程国际学会(ISNSCE) \href{https://isnsce.org/awards/the-dna-student-awards/}{最佳学生报告奖},第25届DNA计算与分子编程国际会议(DNA25),2019年8月。
  \item 优秀教学奖,爱荷华州立大学,2017年。
\end{etaremune}

\section{特邀报告}
\textit{Computing Real Numbers with Large-Population Protocols},Swarthmore学院,2023年10月。 \\
\textit{Some Thoughts on Normality, Algorithmic Randomness, and Analog Computing},第五届南京大学青年论坛,2020年5月(远程)。 \\
\textit{Asymptotic Divergences and Strong Dichotomy},爱荷华信息、复杂性与逻辑研讨会(ICICL),2019年春季。

\section{学术报告}
\textit{Computing Real Numbers with Large-Population Protocols Having a Continuum of Equilibria},DNA 28,2022年8月。 \\
\textit{Real-Time Equivalence of Chemical Reaction Networks and Analog Computers},DNA 25,2019年8月。 \\
\textit{Real-Time Computability of Real Numbers by Chemical Reaction Networks},第19届逻辑研究生会议,威斯康星州麦迪逊,2018年4月。 \\
\textit{Real-Time Computability of Real Numbers by Chemical Reaction Networks},UCNC 2017。

\section{教学经验}

\textbf{伊利诺伊大学斯普林菲尔德分校}
\begin{itemize}
  \item[] CSC 570F -- 研究生算法与应用 \hfill {2023年春季}
  \item[] CSC 302 -- 离散结构 \hfill {2020年秋季至今}
  \item[] CSC 482 -- 算法与计算理论 \hfill {2020年秋季至今}
\end{itemize}

\textbf{勒莫因学院}
\begin{itemize}
\item[] CSC 175 -- 算法与程序设计导论 \hfill {2019年秋季}
\item[] CSC 170 -- Java导论(无编程经验) \hfill {2020年春季}
\item[] CSC 176 -- Java导论(第二门编程课程) \hfill {2020年春季}
\item[] CSC 276 -- 使用Java的面向对象设计 \hfill {2020年春季}
\end{itemize}

\textbf{爱荷华州立大学(助教)}
\begin{itemize}
\item[] COM S 531 -- 计算理论(研究生) \hfill {2014年春季,2016年春季}
\item[] COM S 511 -- 算法设计与分析(研究生) \hfill {2014年秋季,2015年秋季,2017年秋季}
\item[] COM S 331 -- 计算理论 \hfill {2016年秋季,2019年春季}
\item[] COM S 311 -- 算法设计 \hfill {2015年夏季,2016年夏季,2018年秋季}
\item[] COM S 330 -- 离散数学结构 \hfill {2014年春季}
\item[] COM S 252 -- 操作系统导论 \hfill {2013年秋季}
\end{itemize}

\section{本科生研究指导}
精选本科生研究项目(完整列表见\href{https://www.xianghuang.org/mypages/research.html}{xianghuang.org}):
\begin{innerlist}
\item Rachel Huls(2021--2022年):大规模群体协议计算能力研究,发表于DNA 28,\href{https://arxiv.org/abs/2206.06594}{论文链接}。
\item Anish Sinha(2022--2023年):并发B-Link树。获得UIS STARS 2023最佳研究成果奖。
\item Jonathan Miller(2023年):大整数乘法算法研究。\href{https://www.xianghuang.org/mypages/files/Mul_Survey.pdf}{综述论文链接}。
\end{innerlist}

共指导11名本科生(2021年至今)进行理论计算机科学研究。

\end{document}
