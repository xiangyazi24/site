%%%%%%%%%%%%%%%%%%%%%%%%%%%%%%%%%%%%%%%%%%%%%%%%%%%%%%%%%%%%%%%%%%%%%%%%
%%%%%%%%%%%%%%%%%%%%%% Simple LaTeX CV Template %%%%%%%%%%%%%%%%%%%%%%%%
%%%%%%%%%%%%%%%%%%%%%%%%%%%%%%%%%%%%%%%%%%%%%%%%%%%%%%%%%%%%%%%%%%%%%%%%

%%%%%%%%%%%%%%%%%%%%%%%%%%%%%%%%%%%%%%%%%%%%%%%%%%%%%%%%%%%%%%%%%%%%%%%%
%% NOTE: If you find that it says                                     %%
%%                                                                    %%
%%                           1 of ??                                  %%
%%                                                                    %%
%% at the bottom of your first page, this means that the AUX file     %%
%% was not available when you ran LaTeX on this source. Simply RERUN  %%
%% LaTeX to get the ``??'' replaced with the number of the last page  %%
%% of the document. The AUX file will be generated on the first run   %%
%% of LaTeX and used on the second run to fill in all of the          %%
%% references.                                                        %%
%%%%%%%%%%%%%%%%%%%%%%%%%%%%%%%%%%%%%%%%%%%%%%%%%%%%%%%%%%%%%%%%%%%%%%%%

%%%%%%%%%%%%%%%%%%%%%%%%%%%% Document Setup %%%%%%%%%%%%%%%%%%%%%%%%%%%%

% Don't like 10pt? Try 11pt or 12pt
\documentclass[10pt]{article}
% !TeX program  = XeLaTeX
% The automated optical recognition software used to digitize resume
% information works best with fonts that do not have serifs. This
% command uses a sans serif font throughout. Uncomment both lines (or at
% least the second) to restore a Roman font (i.e., a font with serifs).
%\usepackage{times}
%\renewcommand{\familydefault}{\sfdefault}

% This is a helpful package that puts math inside length specifications
\usepackage{calc}
\usepackage{comment}
\usepackage{setspace}
\usepackage{graphicx}
\usepackage[UTF8]{ctex} % for Chinese support
\usepackage{etaremune} % reverse enumberate
\graphicspath{ {images/}} %upload your signature to this file

% Simpler bibsection for CV sections
% (thanks to natbib for inspiration)
\makeatletter
\newlength{\bibhang}
\setlength{\bibhang}{1em} %1em}
\newlength{\bibsep}
 {\@listi \global\bibsep\itemsep \global\advance\bibsep by\parsep}
\newenvironment{bibsection}%
        {\begin{enumerate}{}{%
%        {\begin{list}{}{%
       \setlength{\leftmargin}{\bibhang}%
       \setlength{\itemindent}{-\leftmargin}%
       \setlength{\itemsep}{\bibsep}%
       \setlength{\parsep}{\z@}%
        \setlength{\partopsep}{0pt}%
        \setlength{\topsep}{0pt}}}
        {\end{enumerate}\vspace{-.6\baselineskip}}
%        {\end{list}\vspace{-.6\baselineskip}}
\makeatother

% Layout: Puts the section titles on left side of page
\reversemarginpar

%
%         PAPER SIZE, PAGE NUMBER, AND DOCUMENT LAYOUT NOTES:
%
% The next \usepackage line changes the layout for CV style section
% headings as marginal notes. It also sets up the paper size as either
% letter or A4. By default, letter was used. If A4 paper is desired,
% comment out the letterpaper lines and uncomment the a4paper lines.
%
% As you can see, the margin widths and section title widths can be
% easily adjusted.
%
% ALSO: Notice that the includefoot option can be commented OUT in order
% to put the PAGE NUMBER *IN* the bottom margin. This will make the
% effective text area larger.
%
% IF YOU WISH TO REMOVE THE ``of LASTPAGE'' next to each page number,
% see the note about the +LP and -LP lines below. Comment out the +LP
% and uncomment the -LP.
%
% IF YOU WISH TO REMOVE PAGE NUMBERS, be sure that the includefoot line
% is uncommented and ALSO uncomment the \pagestyle{empty} a few lines
% below.
%

%% Use these lines for letter-sized paper
\usepackage[paper=letterpaper,
            %includefoot, % Uncomment to put page number above margin
            marginparwidth=1.2in,     % Length of section titles
            marginparsep=.05in,       % Space between titles and text
            margin=1in,               % 1 inch margins
            includemp]{geometry}

%% Use these lines for A4-sized paper
%\usepackage[paper=a4paper,
%            %includefoot, % Uncomment to put page number above margin
%            marginparwidth=30.5mm,    % Length of section titles
%            marginparsep=1.5mm,       % Space between titles and text
%            margin=25mm,              % 25mm margins
%            includemp]{geometry}

%% More layout: Get rid of indenting throughout entire document
\setlength{\parindent}{0in}

\usepackage[shortlabels]{enumitem}
\usepackage{lastpage} % for the page number
%% Reference the last page in the page number
%
% NOTE: comment the +LP line and uncomment the -LP line to have page
%       numbers without the ``of ##'' last page reference)
%
% NOTE: uncomment the \pagestyle{empty} line to get rid of all page
%       numbers (make sure includefoot is commented out above)
%
\usepackage{fancyhdr,lastpage}
\pagestyle{fancy}
%\pagestyle{empty}      % Uncomment this to get rid of page numbers
\fancyhf{}\renewcommand{\headrulewidth}{0pt}
\fancyfootoffset{\marginparsep+\marginparwidth}
\newlength{\footpageshift}
\setlength{\footpageshift}
          {0.5\textwidth+0.5\marginparsep+0.5\marginparwidth-2in}
\lfoot{\hspace{\footpageshift}%
       \parbox{4in}{\, \hfill %
               %     \arabic{page} of \protect\pageref*{LastPage} % +LP
%                    \arabic{page}                               % -LP
                    \hfill \,}}

% Finally, give us PDF bookmarks
\usepackage{color,hyperref}
\definecolor{darkblue}{rgb}{0.0,0.0,0.3}
\hypersetup{colorlinks,breaklinks,
            linkcolor=darkblue,urlcolor=darkblue,
            anchorcolor=darkblue,citecolor=darkblue}

%%%%%%%%%%%%%%%%%%%%%%%% End Document Setup %%%%%%%%%%%%%%%%%%%%%%%%%%%%


%%%%%%%%%%%%%%%%%%%%%%%%%%% Helper Commands %%%%%%%%%%%%%%%%%%%%%%%%%%%%

% The title (name) with a horizontal rule under it
% (optional argument typesets an object right-justified across from name
%  as well)
%
% Usage: \makeheading{name}
%        OR
%        \makeheading[right_object]{name}
%
% Place at top of document. It should be the first thing.
% If ``right_object'' is provided in the square-braced optional
% argument, it will be right justified on the same line as ``name'' at
% the top of the CV. For example:
%
%       \makeheading[\emph{Curriculum vitae}]{Your Name}
%
% will put an emphasized ``Curriculum vitae'' at the top of the document
% as a title. Likewise, a picture could be included:
%
%   \makeheading[\includegraphics[height=1.5in]{my_picutre}]{Your Name}
%
% the picture will be flush right across from the name.
\newcommand{\makeheading}[2][]%
        {\hspace*{-\marginparsep minus \marginparwidth}%
         \begin{minipage}[t]{\textwidth+\marginparwidth+\marginparsep}%
             {\large \bfseries #2 \hfill #1}\\[-0.15\baselineskip]%
                 \rule{\columnwidth}{1pt}%
         \end{minipage}}

% The section headings
%
% Usage: \section{section name}
\renewcommand{\section}[1]{\pagebreak[3]%
    \hyphenpenalty=10000%
    \vspace{1.3\baselineskip}%
    \phantomsection\addcontentsline{toc}{section}{#1}%
    \noindent\llap{\scshape\smash{\parbox[t]{\marginparwidth}{\raggedright #1}}}%
    \vspace{-\baselineskip}\par}

% An itemize-style list with lots of space between items
\newenvironment{outerlist}[1][\enskip\textbullet]%
        {\begin{itemize}[#1,leftmargin=*]}{\end{itemize}%
         \vspace{-.6\baselineskip}}

% An environment IDENTICAL to outerlist that has better pre-list spacing
% when used as the first thing in a \section
\newenvironment{lonelist}[1][\enskip\textbullet]%
        {\begin{list}{#1}{%
        \setlength{\partopsep}{0pt}%
        \setlength{\topsep}{0pt}}}
        {\end{list}\vspace{-.6\baselineskip}}

% An itemize-style list with little space between items
\newenvironment{innerlist}[1][\enskip\textbullet]%
        {\begin{itemize}[#1,leftmargin=*,parsep=0pt,itemsep=0pt,topsep=0pt,partopsep=0pt]}
        {\end{itemize}}

% An environment IDENTICAL to innerlist that has better pre-list spacing
% when used as the first thing in a \section
\newenvironment{loneinnerlist}[1][\enskip\textbullet]%
        {\begin{itemize}[#1,leftmargin=*,parsep=0pt,itemsep=0pt,topsep=0pt,partopsep=0pt]}
        {\end{itemize}\vspace{-.6\baselineskip}}

% To add some paragraph space between lines.
% This also tells LaTeX to preferably break a page on one of these gaps
% if there is a needed pagebreak nearby.
\newcommand{\blankline}{\quad\pagebreak[3]}
\newcommand{\halfblankline}{\quad\vspace{-0.5\baselineskip}\pagebreak[3]}

% Uses hyperref to link DOI
\newcommand\doilink[1]{\href{http://dx.doi.org/#1}{#1}}
\newcommand\doi[1]{doi:\doilink{#1}}

% For \url{SOME_URL}, links SOME_URL to the url SOME_URL
\providecommand*\url[1]{\href{#1}{#1}}
% Same as above, but pretty-prints SOME_URL in teletype fixed-width font
\renewcommand*\url[1]{\href{#1}{\texttt{#1}}}

% For \email{ADDRESS}, links ADDRESS to the url mailto:ADDRESS
\providecommand*\email[1]{\href{mailto:#1}{#1}}
% Same as above, but pretty-prints ADDRESS in teletype fixed-width font
%\renewcommand*\email[1]{\href{mailto:#1}{\texttt{#1}}}

%\providecommand\BibTeX{{\rm B\kern-.05em{\sc i\kern-.025em b}\kern-.08em
%    T\kern-.1667em\lower.7ex\hbox{E}\kern-.125emX}}
%\providecommand\BibTeX{{\rm B\kern-.05em{\sc i\kern-.025em b}\kern-.08em
%    \TeX}}
\providecommand\BibTeX{{B\kern-.05em{\sc i\kern-.025em b}\kern-.08em
    \TeX}}
\providecommand\Matlab{\textsc{Matlab}}

%Skype information - include your Skype name for a link to add you on Skype
\newcommand*{\Skype}{\href{skype:xiangyazi24?add}{xiangyazi24}}
\newcommand{\Absender}[1][\normalsize]{\Skype}
%%%%%%%%%%%%%%%%%%%%%%%% End Helper Commands %%%%%%%%%%%%%%%%%%%%%%%%%%%

%%%%%%%%%%%%%%%%%%%%%%%%% Begin CV Document %%%%%%%%%%%%%%%%%%%%%%%%%%%%

\begin{document}
\makeheading{\Large{Huang, Xiang} \hfill University of Illinois Springfield}

\section{Contact Information}

% NOTE: Mind where the & separators and \\ breaks are in the following
%       table.
%
% ALSO: \rcollength is the width of the right column of the table
%       (adjust it to your liking; default is 1.85in).
%
\newlength{\rcollength}\setlength{\rcollength}{1.4in}%
%
\begin{center}
\begin{tabular}{l l}
 Department of Computer Science & \hspace{1in} \href{mailto:xhuan5@uis.edu}{xhuan5@uis.edu} \\
   3115 UHB, One University Plaza &  \hspace{1in} Phone: +1 (217) 206-8336 \\
  Springfield, IL 62703-5407, USA        &
\end{tabular}
\end{center}
%\section{Objective}

%Insert text here if you want to
%\begin{innerlist}
%\item More information and auxiliary documents can be found at\\\url{http://www.tedpavlic.com/facjobsearch/}
%\end{innerlist}
\section{Current Position}
Assistant Professor, University of Illinois Springfield, Springfield, IL, US. (August, 2020  to present)

\section{Personal Website}
\href{https://www.xianghuang.org}{\textbf{xianghuang.org}}

\section{Research Interests}

Algorithmic Information Theory, Analog Computing, Molecular Programming, and Theoretical Foundations.

\section{Visiting Position:}
Visiting Assistant Professor, Le Moyne College, Syracuse, NY, US. (Sept, 2019  to June, 2019 )

\section{Education}

\href{http://www.iastate.edu/}{\textbf{Iowa State University}},
IA, US.
\begin{outerlist}

\item[]  {Ph.D. in Computer Science, 2020},
        \begin{innerlist}
        \item Thesis: \emph{Chemical Reaction Networks: Computability, Complexity, and Randomness}
        \item Advisor:
              \href{http://jacklutz.com}
                   {Professor Jack H. Lutz}
        \end{innerlist}
\end{outerlist}
\href{http://english.is.cas.cn/}{\textbf{Institute of Software, Chinese Academy of Sciences,}}  Beijing, China.
\begin{outerlist}
\item[]  {Computer Science},
             2009.09 - 2012.06.
        \begin{innerlist}
        \item Topic: \emph{Model Checking, Formal Methods, Automata Theory}
        \end{innerlist}
\end{outerlist}
\vspace{.1in}
\href{http://www.nju.edu.cn}{\textbf{Nanjing University, }} Nanjing, China.
\begin{outerlist}
\item[]{B.E. in Software Engineering} , 2005.09 - 2009.06.
\end{outerlist}

\section{Grant Supports}

External Support:
\begin{etaremune}
   \item Principal investigator: \textit{Towards A Hierarchy of Real Numbers Computable by CRN}, \$400K, Department of Energy EXPRESS grant, 2023-2025.
\end{etaremune}
U of I Springfield  or U of I System internal supports:
\begin{etaremune}
    \item National Taiwan University‐University of Illinois System Travel Grants Program, \$5,000, 2024.
    \item Competitive Scholarly Research Grant, \$5,000, 2023-2024.
    \item Grant Writing Mentorship Award, \$1,500, 2022-2023.
    \item Leadership Lived Experience (LLE) student employment initiative, \$4,000, 2022.
\end{etaremune}
\section{Journal Publications}

\begin{etaremune}
    \item Xiang Huang, Jack H. Lutz, Elvira Mayordomo, and Donald M. Stull, Asymptotic divergences and strong dichotomy, IEEE Transactions on Information Theory 67 (2021), pp. 6296-6305.
    \item Xiang Huang, Titus H. Klinge, James I. Lathrop, Xiaoyuan Li and Jack H. Lutz: Real-Time Computability of Real Numbers by Chemical Reaction Networks. \textit{Volume 18, Issue 1, pp 63-73, Natural Computing (2019).  \textbf{(invited paper).}}
\end{etaremune}
%\section{Journal Submissions Under Review}
%\begin{itemize}
%\item Xiang Huang, Titus H. Klinge, James I. Lathrop. Equivalence of Real-Time Computable Numbers in Analog Models. Submitted.
%\item Xiang Huang, Jack H. Lutz, and Andrei N. Migunov. Algorithmic Randomness in Continuous-Time Markov Chains. Submitted.
%\end{itemize}

\section{Conference Publications}
\begin{etaremune}
  \item Xiang Huang and Rachel Huls. Computing Real Numbers with Large-Population Protocols Having a Continuum of Equilibria. The 28th International Conference on DNA Computing and Molecular Programming (DNA 28, Albuquerque, NM, Aug 8-12, 2022).
  \item Xiang Huang, Jack H. Lutz, Elvira Mayordomo, and Donald M. Stull. Asymptotic divergences and strong dichotomy, Proceedings of the Thirty-seventh Symposium on Theoretical Aspects of Computer Science (STACS 2020, Montpellier, France, March 10-13, 2020).
  \item Xiang Huang, Jack H. Lutz, and Andrei N. Migunov. Algorithmic Randomness in Continuous-Time Markov Chains, 2019. In Proceedings of the 57th Annual Allerton Conference on Communication, Control, and Computing.
  \item Xiang Huang, Titus H. Klinge, James I. Lathrop. Real-Time Equivalence of Chemical Reaction Networks and Analog Computers. In: Thachuk C., Liu Y. (eds) DNA Computing and Molecular Programming. DNA 2019. Lecture Notes in Computer Science, vol 11648. Springer, Cham.
  \item Xiang Huang, Titus H. Klinge, James I. Lathrop,  Xiaoyuan Li and Jack H. Lutz. Real-Time Computability of Real Numbers by Chemical Reaction Networks. \textit{In Proceedings of the 16th International Conference on Unconventional Computation and Natural Computation (UCNC)} , June 2017, pp. 29-40.
  \item Xiang Huang and Donald.~M. Stull.  Polynomial Space Randomness in Analysis.  \textit{In Proceedings of the 41st International Symposium on Mathematical Foundations of Computer Science (MFCS)} , August 2016:86:1-86:13.
\end{etaremune}
\section{Peer-Reviewed Workshop Paper/Extended Abstract}
\begin{etaremune}
  \item Xiang Huang and Andrei N. Migunov. A General Purpose Analog Computer to Population Protocol Compiler. The 3rd 3rd Annual Compiler Frontiers Workshop, May 9, 2024.
\end{etaremune}
\section{Awards}
\begin{etaremune}
  \item The International Society for Nanoscale Science, Computation and Engineering
(ISNSCE) \href{https://isnsce.org/awards/the-dna-student-awards/}{Best Student Presentation Award}, at 25th International Conference on DNA Computing and
Molecular Programming (DNA25), August, 2019.
  \item Teaching Excellence Award, 2017, Iowa State University.
\end{etaremune}

%Xiang Huang, Jack H. Lutz, Elvira Mayordomo, and Donald Stull. Asymptotic Divergences and Strong Dichotomy, 2019. (under review) \\
%%%%%%%%%%%%%%%%% REFERENCES %%%%%%%%%%%%%%%%%%%%%%%%%%
% The reference section has links to your references' websites and email addresses.

\section{Invited Talks}
\textit{Computing Real Numbers with Large-Population Protocols}, Drake University,  Oct 27, 2023.\\
\textit{Some Thoughts on Normality, Algorithmic Randomness, and Analog Computing}, the Fifth Nanjing University Youth Forum, May, 2020. (Remote)

\section{Contributed Talks}
\textit{Computing Real Numbers with Large-Population Protocols Having a Continuum of Equilibria.}, DNA 28, August, 2022.\\
\textit{Real-Time Equivalence of Chemical Reaction Networks and Analog Computers}, DNA 25, August, 2019.\\
\textit{Asymptotic Divergences and Strong Dichotomy}, Iowa Colloquium on Information, Complexity, and Logic (ICICL), Spring 2019.\\
\textit{Some Thoughts on Normality, Algorithmic Randomness, and Analog Computing}, Swarthmore College, Swarthmore, PA, March 2019. \\
\textit{Real-Time Computability of Real Numbers by Chemical Reaction Networks}, the 19th Graduate Student Conference in Logic, Madison, WI, April 2018 \\
\textit{Real-Time Computability of Real Numbers by Chemical Reaction Networks}, UCNC 2017.\\

\section{Teaching Experience}

\textbf{At UIS}
\begin{itemize}
  \item[] CSC 570F - Graduate Algorithms and Applications \hfill {Spring 2023}
  \item[] CSC 302 - Discrete Structures \hfill {Fall 2020 to present}
  \item[] CSC 482 - Algorithms and Theorey of Computation \hfill {Fall 2020 to present}
\end{itemize}
\textbf{As instructor at Le Moyne College}

\begin{itemize}
\item[] CSC 175 - Introduction to Algorithms and Program Design.\hfill {Fall 2019}
\item[] CSC 170 - Java Introduction (no prior programming experience) \hfill {Spring 2020}
\item[] CSC 176 - Java Introduction (as a second programming course)  \hfill {Spring 2020}
\item[] CSC 276 - Object Oriented Design Using Java \hfill{Spring 2020}
\end{itemize}
\textbf{As teaching Assistant at Iowa State}
\begin{itemize}
\item[] COM S 531 - Theory of Computation (Grad)\hfill {Spring 2014 and 2016}
\item[] COM S 511 - Algorithm Design and Analysis (Grad)\hfill {Fall 2014, 2015, and 2017}
\item[] COM S 331 - Theory of Computation \hfill {Fall 2016 and Spring 2019}
\item[] COM S 311 - Algorithm Design \hfill {Summer 2015, 2016, and Fall 2018 }
\item[] COM S 330 - Discrete Mathematical Structures \hfill {Spring 2014}
\item[] COM S 252 - Introduction to Operating Systems\hfill {Fall 2013}
\end{itemize}
%\vspace{4mm}
%\begin{tabular}{@{} l }
%   \href{https://www.cs.iastate.edu/people/simanta-mitra}{Simanta Mitra }(for teaching) \\
%   Senior Lecturer \\
% Department of Computer Science\\
% Iowa State University\\
% \small{\href{mailto: smitra@iastate.edu}{smitra@iastate.edu}}
%\end{tabular}

\end{document}

%%%%%%%%%%%%%%%%%%%%%%%%%% End CV Document %%%%%%%%%%%%%%%%%%%%%%%%%%%%%

%----------------------------------------------------------------------%
% The following is copyright and licensing information for
% redistribution of this LaTeX source code; it also includes a liability
% statement. If this source code is not being redistributed to others,
% it may be omitted. It has no effect on the function of the above code.
%----------------------------------------------------------------------%
% Copyright (c) 2007, 2008, 2009, 2010, 2011 by Theodore P. Pavlic
%
% Unless otherwise expressly stated, this work is licensed under the
% Creative Commons Attribution-Noncommercial 3.0 United States License. To
% view a copy of this license, visit
% http://creativecommons.org/licenses/by-nc/3.0/us/ or send a letter to
% Creative Commons, 171 Second Street, Suite 300, San Francisco,
% California, 94105, USA.
%
% THE SOFTWARE IS PROVIDED "AS IS", WITHOUT WARRANTY OF ANY KIND, EXPRESS
% OR IMPLIED, INCLUDING BUT NOT LIMITED TO THE WARRANTIES OF
% MERCHANTABILITY, FITNESS FOR A PARTICULAR PURPOSE AND NONINFRINGEMENT.
% IN NO EVENT SHALL THE AUTHORS OR COPYRIGHT HOLDERS BE LIABLE FOR ANY
% CLAIM, DAMAGES OR OTHER LIABILITY, WHETHER IN AN ACTION OF CONTRACT,
% TORT OR OTHERWISE, ARISING FROM, OUT OF OR IN CONNECTION WITH THE
% SOFTWARE OR THE USE OR OTHER DEALINGS IN THE SOFTWARE.
%----------------------------------------------------------------------%
