%=====================ComS 511 LaTeX template, following the CMU 02-713 template================
%
% You don't need to use LaTeX or this template, but you must turn your homework in as
% a typeset PDF somehow.
%
% How to use:
%    1. Update your information in section "A" below
%    2. Write your answers in section "B" below. Precede answers for all
%       parts of a question with the command "\question{n}{desc}" where n is
%       the question number and "desc" is a short, one-line description of
%       the problem. There is no need to restate the problem.
%    3. If a question has multiple parts, precede the answer to part x with the
%       command "\part{x}".
%    4. If a problem asks you to design an algorithm, use the commands
%       \algorithm, \correctness, \runtime to precede your discussion of the
%       description of the algorithm, its correctness, and its running time, respectively.
%    5. You can include graphics by using the command \includegraphics{FILENAME}
%

\documentclass[11pt]{article}
\usepackage{amsmath,amssymb,amsthm}
\usepackage{times}
\usepackage{zi4}
\usepackage{graphicx}
\usepackage[margin=1in]{geometry}
\usepackage{fancyhdr}
\usepackage{hyperref}
\setlength{\parindent}{0pt}
\setlength{\parskip}{5pt plus 1pt}
\setlength{\headheight}{13.6pt}
\newcommand\question[2]{\vspace{.25in}\hrule\textbf{#1: #2}\vspace{.5em}\hrule\vspace{.10in}}
\renewcommand\part[1]{\vspace{.10in}\textbf{(#1)}}
\newcommand\algorithm{\vspace{.10in}\textbf{Algorithm: }}
\newcommand\correctness{\vspace{.10in}\textbf{Correctness: }}
\newcommand\runtime{\vspace{.10in}\textbf{Running time: }}
\newcommand\construction{\vspace{.10in}\textbf{Construction:}}
\newcommand\solution{\vspace{.10in}\textbf{Solution: }}

%%%%% theorem styles
\newtheorem{theorem}{Theorem}
\newtheorem{acknowledgement}[theorem]{Acknowledgement}
%\newtheorem{algorithm}[theorem]{Algorithm}
\newtheorem{axiom}{Axiom}
\newtheorem{case}[theorem]{Case}
\newtheorem{claim}[theorem]{Claim}
\newtheorem{conclusion}[theorem]{Conclusion}
\newtheorem{condition}[theorem]{Condition}
\newtheorem{conjecture}[theorem]{Conjecture}
\newtheorem{corollary}[theorem]{Corollary}
\newtheorem{criterion}[theorem]{Criterion}
\newtheorem{definition}[theorem]{Definition}
\newtheorem{example}[theorem]{Example}
\newtheorem{exercise}[theorem]{Exercise}
\newtheorem{lemma}[theorem]{Lemma}
\newtheorem{notation}[theorem]{Notation}
\newtheorem{problem}[theorem]{Problem}
\newtheorem{proposition}[theorem]{Proposition}
\newtheorem{remark}[theorem]{Remark}
%\newtheorem{solution}[theorem]{Solution}
\newtheorem{summary}[theorem]{Summary}
%\newenvironment{proof}[1][Proof]{\textbf{#1.} }{\ \rule{0.5em}{0.5em}}

\newcommand{\Q}{\mathbb{Q}}
\newcommand{\R}{\mathbb{R}}
\newcommand{\Z}{\mathbb{Z}}
\newcommand{\N}{\mathbb{N}}
%%% end of theorem styles

%%%===========algorithm
%% algorithm
\usepackage{listings}
\usepackage{algorithm}
\usepackage[noend]{algpseudocode}
\usepackage{etoolbox}

\makeatletter
% start with some helper code
% This is the vertical rule that is inserted
\newcommand*{\algrule}[1][\algorithmicindent]{%
  \makebox[#1][l]{%
    \hspace*{.2em}% <------------- This is where the rule starts from
    \vrule height .75\baselineskip depth .25\baselineskip
  }
}

\newcount\ALG@printindent@tempcnta
\def\ALG@printindent{%
    \ifnum \theALG@nested>0% is there anything to print
    \ifx\ALG@text\ALG@x@notext% is this an end group without any text?
    % do nothing
    \else
    \unskip
    % draw a rule for each indent level
    \ALG@printindent@tempcnta=1
    \loop
    \algrule[\csname ALG@ind@\the\ALG@printindent@tempcnta\endcsname]%
    \advance \ALG@printindent@tempcnta 1
    \ifnum \ALG@printindent@tempcnta<\numexpr\theALG@nested+1\relax
    \repeat
    \fi
    \fi
}
% the following line injects our new indent handling code in place of the default spacing
\patchcmd{\ALG@doentity}{\noindent\hskip\ALG@tlm}{\ALG@printindent}{}{\errmessage{failed to patch}}
\patchcmd{\ALG@doentity}{\item[]\nointerlineskip}{}{}{} % no spurious vertical space
% end vertical rule patch for algorithmicx
\makeatother
%%% ==========end of algorithm
\pagestyle{fancyplain}
\lhead{\textbf{\NAME\ (\UISID)}}
\chead{\textbf{HW\HWNUM}}
\rhead{\textsc{CSC 302}, \today}
\begin{document}\raggedright
%Section A==============Change the values below to match your information==================
\newcommand\NAME{Xiang Huang}  % your name
\newcommand\UISID{\texttt{UISid}}     % your UIS id
\newcommand\HWNUM{12}              % the homework number
%Section B==============Put your answers to the questions below here=======================



% no need to restate the problem --- the graders know which problem is which,
% but replacing "The First Problem" with a short phrase will help you remember
% which problem this is when you read over your homeworks to study.

\textbf{Reading:}

\question{1}{Trivial Equivalent Relation (10 pts) *}
Consider all the binary strings of length 3, or S=$\{0,1\}^3$ If we say for $s, t \in S$, $s R t$ if and only if $s=t$. Then what are all the equivalent classes?

\question{2}{Binary String Equivalence (10 pts) **}
Let $S=\{0,1\}^4$, that is, all the binary strings of length 4. Let relation $R$ be defined as for $s,t\in S$,
$s R t$ if and only if $s$ and $t$ has the same number of zeros.

\begin{enumerate}
  \item[(a)] Show $R$ is an equivalent relation.
  \item[(b)] List all the equivalent classes. Please recall that a equivalent class is a subset of S, and the equivalent classes form a partition of $S$.
\end{enumerate}


\question{3}{Binary String Equivalence II (10 pts) **}
Let $S=\{0,1\}^4$,that is, all the binary strings of length 4. For $s, t$ be a string in $S$, we define $sRt$ if and only $t$ can be obtained from $s$ by rotatation. For example, we can get $1011$ by rotating $0111$ to the right.

\begin{enumerate}
  \item[(a)] Show $R$ is an equivalent relation.
  \item[(b)] List all the equivalent classes.
\end{enumerate}

\question{4}{Counting Partisions or Equivalent Class (10 pts) *** }

Let $p(n)$ denote the number of different equivalence
relations on a set with $n$ elements (or equivalently, the number of partitions of a set with n elements). Show that $p(n)$ satisfies the recurrence relation
 \[
   p(n)=\sum_0^{n-1} \binom{n-1}{j} p(n-1-j),
 \]
 and the initial
condition p(0) = 1. \\

Hint: Use recursive thinking. Consider the last element $a_n$, and the rest of $n-1$ elements. Is $a_n$ going to be in the same partition with any of the rest of $n-1$ elements? How many?
\question{5}{Seven Bridges of K\"onigsberg}

 Read/watch the following materials and write a report on Euler's analysis and rewrite his proof of his following theorem.
 \begin{enumerate}
   \item \href{https://en.wikipedia.org/wiki/Seven_Bridges_of_K%C3%B6nigsberg}{Seven Bridges of K\"onigsberg}
   \item \href{https://www.youtube.com/watch?v=dSK5jTEe-AM}{Euler Paths \& the 7 Bridges of Konigsberg }
 \end{enumerate}

 \begin{theorem}{Euler's Theorem:}
    A connected graph has an Euler cycle if and only if every vertex has even degree.
 \end{theorem}

 You might want to read the Wiki page for \href{https://en.wikipedia.org/wiki/Eulerian_path}{Eulerian Path} for more information.

 In case you can not open the link, here are the list of links:

 \begin{verbatim}
  1. https://en.wikipedia.org/wiki/Seven_Bridges_of_K%C3%B6nigsberg
  2. https://www.youtube.com/watch?v=dSK5jTEe-AM
  3. https://en.wikipedia.org/wiki/Eulerian_path
 \end{verbatim}




\end{document}
