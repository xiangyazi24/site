%=====================ComS 511 LaTeX template, following the CMU 02-713 template================
%
% You don't need to use LaTeX or this template, but you must turn your homework in as
% a typeset PDF somehow.
%
% How to use:
%    1. Update your information in section "A" below
%    2. Write your answers in section "B" below. Precede answers for all
%       parts of a question with the command "\question{n}{desc}" where n is
%       the question number and "desc" is a short, one-line description of
%       the problem. There is no need to restate the problem.
%    3. If a question has multiple parts, precede the answer to part x with the
%       command "\part{x}".
%    4. If a problem asks you to design an algorithm, use the commands
%       \algorithm, \correctness, \runtime to precede your discussion of the
%       description of the algorithm, its correctness, and its running time, respectively.
%    5. You can include graphics by using the command \includegraphics{FILENAME}
%

\documentclass[11pt]{article}
\usepackage{amsmath,amssymb,amsthm}
\usepackage{times}
\usepackage{zi4}
\usepackage{graphicx}
\usepackage[margin=1in]{geometry}
\usepackage{fancyhdr}
\usepackage{listings} % for code listing
\usepackage{xcolor}  % for the colors


%% listing style
\definecolor{codegreen}{rgb}{0,0.6,0}
\definecolor{codegray}{rgb}{0.5,0.5,0.5}
\definecolor{codepurple}{rgb}{0.58,0,0.82}
\definecolor{backcolour}{rgb}{0.95,0.95,0.92}

\lstdefinestyle{mystyle}{
    backgroundcolor=\color{backcolour},
    commentstyle=\color{codegreen},
    keywordstyle=\color{magenta},
    numberstyle=\tiny\color{codegray},
    stringstyle=\color{codepurple},
    basicstyle=\ttfamily\footnotesize,
    breakatwhitespace=false,
    breaklines=true,
    captionpos=b,
    keepspaces=true,
    numbers=left,
    numbersep=5pt,
    showspaces=false,
    showstringspaces=false,
    showtabs=false,
    tabsize=2
}

\lstset{style=mystyle}


%% end of listing style

\setlength{\parindent}{0pt}
\setlength{\parskip}{5pt plus 1pt}
\setlength{\headheight}{13.6pt}
\newcommand\question[2]{\vspace{.25in}\hrule\textbf{#1: #2}\vspace{.5em}\hrule\vspace{.10in}}
\renewcommand\part[1]{\vspace{.10in}\textbf{(#1)}}
\newcommand\algorithm{\vspace{.10in}\textbf{Algorithm: }}
\newcommand\correctness{\vspace{.10in}\textbf{Correctness: }}
\newcommand\runtime{\vspace{.10in}\textbf{Running time: }}
\newcommand\construction{\vspace{.10in}\textbf{Construction:}}
\newcommand\solution{\vspace{.10in}\textbf{Solution: }}


%%%%% theorem styles
\newtheorem{theorem}{Theorem}
\newtheorem{acknowledgement}[theorem]{Acknowledgement}
%\newtheorem{algorithm}[theorem]{Algorithm}
\newtheorem{axiom}{Axiom}
\newtheorem{case}[theorem]{Case}
\newtheorem{claim}[theorem]{Claim}
\newtheorem{conclusion}[theorem]{Conclusion}
\newtheorem{condition}[theorem]{Condition}
\newtheorem{conjecture}[theorem]{Conjecture}
\newtheorem{corollary}[theorem]{Corollary}
\newtheorem{criterion}[theorem]{Criterion}
\newtheorem{definition}[theorem]{Definition}
\newtheorem{example}[theorem]{Example}
\newtheorem{exercise}[theorem]{Exercise}
\newtheorem{lemma}[theorem]{Lemma}
\newtheorem{notation}[theorem]{Notation}
\newtheorem{problem}[theorem]{Problem}
\newtheorem{proposition}[theorem]{Proposition}
\newtheorem{remark}[theorem]{Remark}
%\newtheorem{solution}[theorem]{Solution}
\newtheorem{summary}[theorem]{Summary}
%\newenvironment{proof}[1][Proof]{\textbf{#1.} }{\ \rule{0.5em}{0.5em}}

\newcommand{\Q}{\mathbb{Q}}
\newcommand{\R}{\mathbb{R}}
\newcommand{\C}{\mathbb{C}}
\newcommand{\Z}{\mathbb{Z}}
\newcommand{\N}{\mathbb{N}}
%%% end of theorem styles

%%%===========algorithm
%% algorithm
\usepackage{listings}
\usepackage{algorithm}
\usepackage[noend]{algpseudocode}
\usepackage{etoolbox}

\makeatletter
% start with some helper code
% This is the vertical rule that is inserted
\newcommand*{\algrule}[1][\algorithmicindent]{%
  \makebox[#1][l]{%
    \hspace*{.2em}% <------------- This is where the rule starts from
    \vrule height .75\baselineskip depth .25\baselineskip
  }
}

\newcount\ALG@printindent@tempcnta
\def\ALG@printindent{%
    \ifnum \theALG@nested>0% is there anything to print
    \ifx\ALG@text\ALG@x@notext% is this an end group without any text?
    % do nothing
    \else
    \unskip
    % draw a rule for each indent level
    \ALG@printindent@tempcnta=1
    \loop
    \algrule[\csname ALG@ind@\the\ALG@printindent@tempcnta\endcsname]%
    \advance \ALG@printindent@tempcnta 1
    \ifnum \ALG@printindent@tempcnta<\numexpr\theALG@nested+1\relax
    \repeat
    \fi
    \fi
}
% the following line injects our new indent handling code in place of the default spacing
\patchcmd{\ALG@doentity}{\noindent\hskip\ALG@tlm}{\ALG@printindent}{}{\errmessage{failed to patch}}
\patchcmd{\ALG@doentity}{\item[]\nointerlineskip}{}{}{} % no spurious vertical space
% end vertical rule patch for algorithmicx
\makeatother
%%% ==========end of algorithm
\pagestyle{fancyplain}
\lhead{\textbf{\NAME\ (\UISID)}}
\chead{\textbf{HW\HWNUM}}
\rhead{\textsc{CSC 302}, \today}
\begin{document}\raggedright
%Section A==============Change the values below to match your information==================
\newcommand\NAME{Xiang Huang}  % your name
\newcommand\UISID{\texttt{NetID}}     % your ISU id
\newcommand\HWNUM{1}              % the homework number
%Section B==============Put your answers to the questions below here=======================



% no need to restate the problem --- the graders know which problem is which,
% but replacing "The First Problem" with a short phrase will help you remember
% which problem this is when you read over your homeworks to study.

\textbf{Reading:}
\begin{enumerate}
    \item P\'eter G\'acs's slides page 26 - 52.
    \item Chapter 1 section 1 to 5 \textit{Discrete Mathematics and Its Applications (7th Edition)}.
\end{enumerate}

\textbf{Please write out the intermediate steps when you construct the truth tables.} For example, to do $((p \land q) \to (p \lor q))$, you might want to have a column for $p\land q$ and another for $p\lor q$, before you calculate results for $((p \land q) \to (p \lor q))$.

The enumeration of the truth assignment needs to be in an \textbf{increasing order} in binary encoding, e.g., 000, 001, ..., 111.
\section{Truth Tables}
\question{1}{Construct Truth Tables (10 pts) **}
Construct a truth table for each of these compound propositions. Note that in the following $p\leftrightarrow q$ is defined as $(p\leftarrow q)\land (q\leftarrow p)$. Do you reading to find truth table for the operation.
\begin{enumerate}
    \item $((p \land q) \to (p \lor q))$.
    \item $(p\lor q)\to (p\oplus q)$.
    \item $(p \leftrightarrow q) \oplus (\neg p \leftrightarrow \neg r)$
\end{enumerate}

Here is an example of a truth table by \LaTeX.
\begin{displaymath}
\begin{array}{|c c|c|}
% |c c|c| means that there are three columns in the table and
% a vertical bar ’|’ will be printed on the left and right borders,
% and between the second and the third columns.
% The letter ’c’ means the value will be centered within the column,
% letter ’l’, left-aligned, and ’r’, right-aligned.
p & q & p \land q\\ % Use & to separate the columns
\hline % Put a horizontal line between the table header and the rest.
T & T & T\\
T & F & F\\
F & T & F\\
F & F & F\\
\end{array}
\end{displaymath}
\question{2}{Logical equivalences (10 pts) ***}
You can show the following logical equivalences by truth tables or by equivalence laws.
\begin{enumerate}
    \item Show that $(p \to q) \lor (p \to r)$ and $p \to (q \lor r)$ are logically equivalent.
    \item Show that $\neg p \to (q \to r)$ and $q \to (p \lor r)$ are logically.
equivalent
\end{enumerate}

\question{3}{First Order Logic On Finite Domain (10 pts) **}
Suppose the domain of the propositional function P (x, y)
consists of pairs x and y, where $x\in \{1,2\}$ and $y\in \{1,2,3\}$. Write out the following propositions using disjunctions ($\lor$)
and conjunctions ($\land$). (This problem help you see that in \textbf{finite domain} you can express the same set of statement without quantifiers.)

\begin{enumerate}
    \item[(a)] $\forall x \forall y P(x,y)$,
    \item[(b)] $\exists x \exists y P(x,y)$,
    \item[(c)] $\exists x \forall y P(x,y)$,
    \item[(d)] $\forall y \exists x P(x,y)$.
\end{enumerate}

\question{4}{Nested Quantifiers On Finite Domain and Loops (10 pts) ***}
In Page 58 of \textit{Discrete Mathematics and Its Applications (7th Edition)}  there is a remark ``THINKING OF QUANTIFICATION AS LOOPS''. Suppose now you need to write a code block to check if a first order logic statement with nested quantifier to be true or not. For example, let P(x,y) be the same as Question 1, and x, y now range over 0 to 100 (not including 100), to check if $\forall x\forall y P(x,y)$, you might want to do

\begin{lstlisting}[language=Java, caption=Java example]
boolean AA_P_xy() // A for ``forall'',
{
    for(i=0;i<100; i++)
    {
        for(j=0;j<100; j++)
        {
            if(P(x,y)==false)) return false; // one pair of such (x,y) is enough
        }
    }
    return true;
}
\end{lstlisting}
You tasks, write code block for checking
\begin{enumerate}
    \item $\exists x \exists y P(x,y)$, name your function as ``EE\_P\_xy''.

    \item $\exists x \forall y P(x,y)$, name your function as ``EA\_P\_xy''.

    \item $\forall x \exists y P(x,y)$, name your function as ``AE\_P\_xy''.
\end{enumerate}



\question{5}{contrapositve proof (10 pts)}
Should that if $a + b + c >  2022$, then one of the following must hold:
\begin{enumerate}
     \item[(a)] $a > 2000$.
     \item[(b)] $b> -10 + \sqrt{3}$.
     \item[(c)] $ c> 32 -\sqrt{3}$.
 \end{enumerate}
\end{document}
