%=====================ComS 511 LaTeX template, following the CMU 02-713 template================
%
% You don't need to use LaTeX or this template, but you must turn your homework in as
% a typeset PDF somehow.
%
% How to use:
%    1. Update your information in section "A" below
%    2. Write your answers in section "B" below. Precede answers for all
%       parts of a question with the command "\question{n}{desc}" where n is
%       the question number and "desc" is a short, one-line description of
%       the problem. There is no need to restate the problem.
%    3. If a question has multiple parts, precede the answer to part x with the
%       command "\part{x}".
%    4. If a problem asks you to design an algorithm, use the commands
%       \algorithm, \correctness, \runtime to precede your discussion of the
%       description of the algorithm, its correctness, and its running time, respectively.
%    5. You can include graphics by using the command \includegraphics{FILENAME}
%

\documentclass[11pt]{article}
\usepackage{amsmath,amssymb,amsthm}
\usepackage{times}
\usepackage{zi4}
\usepackage{graphicx}
\usepackage[margin=1in]{geometry}
\usepackage{fancyhdr}
\setlength{\parindent}{0pt}
\setlength{\parskip}{5pt plus 1pt}
\setlength{\headheight}{13.6pt}
\newcommand\question[2]{\vspace{.25in}\hrule\textbf{#1: #2}\vspace{.5em}\hrule\vspace{.10in}}
\renewcommand\part[1]{\vspace{.10in}\textbf{(#1)}}
\newcommand\algorithm{\vspace{.10in}\textbf{Algorithm: }}
\newcommand\correctness{\vspace{.10in}\textbf{Correctness: }}
\newcommand\runtime{\vspace{.10in}\textbf{Running time: }}
\newcommand\construction{\vspace{.10in}\textbf{Construction:}}

%%%%% theorem styles
\newtheorem{theorem}{Theorem}
\newtheorem{acknowledgement}[theorem]{Acknowledgement}
%\newtheorem{algorithm}[theorem]{Algorithm}
\newtheorem{axiom}{Axiom}
\newtheorem{case}[theorem]{Case}
\newtheorem{claim}[theorem]{Claim}
\newtheorem{conclusion}[theorem]{Conclusion}
\newtheorem{condition}[theorem]{Condition}
\newtheorem{conjecture}[theorem]{Conjecture}
\newtheorem{corollary}[theorem]{Corollary}
\newtheorem{criterion}[theorem]{Criterion}
\newtheorem{definition}[theorem]{Definition}
\newtheorem{example}[theorem]{Example}
\newtheorem{exercise}[theorem]{Exercise}
\newtheorem{lemma}[theorem]{Lemma}
\newtheorem{notation}[theorem]{Notation}
\newtheorem{problem}[theorem]{Problem}
\newtheorem{proposition}[theorem]{Proposition}
\newtheorem{remark}[theorem]{Remark}
\newtheorem{solution}[theorem]{Solution}
\newtheorem{summary}[theorem]{Summary}
%\newenvironment{proof}[1][Proof]{\textbf{#1.} }{\ \rule{0.5em}{0.5em}}

\newcommand{\Q}{\mathbb{Q}}
\newcommand{\R}{\mathbb{R}}
\newcommand{\C}{\mathbb{C}}
\newcommand{\Z}{\mathbb{Z}}
\newcommand{\N}{\mathbb{N}}

\usepackage{diagbox} % for the table
\usepackage{hyperref}
%%% end of theorem styles
\pagestyle{fancyplain}
\lhead{\textbf{\NAME\ (\UISID)}}
\chead{\textbf{HW\HWNUM}}
\rhead{\textsc{CSC 302}, \today}
\begin{document}\raggedright
%Section A==============Change the values below to match your information==================
\newcommand\NAME{Xiang Huang}  % your name
\newcommand\UISID{\texttt{UISid}}     % your ISU id
\newcommand\HWNUM{3}              % the homework number
%Section B==============Put your answers to the questions below here=======================

% no need to restate the problem --- the graders know which problem is which,
% but replacing "The First Problem" with a short phrase will help you remember
% which problem this is when you read over your homeworks to study.


\textbf{Reading}: Section 2.3 and 2.5 of the reference book \textit{Discrete Mathematics and Its Applications (7th Edition).}

 Extra video: the video link of a proof of \href{https://www.youtube.com/watch?v=IkoKttTDuxE}{Cantor-Schr\"oder-Berbstein Theorem}

Next lecture we will talk about logic.

\begin{notation}
Recall that we adopt the following convention on notations.
\begin{enumerate}
    \item The notation $\N$ = the set of all natural numbers.
    \item The notation $\Z$ = the set of all whole numbers.
    \item The notation $\Q$ = the set of all rational numbers.
    \item The notation $\R$ = the set of all real numbers.
\end{enumerate}
\end{notation}
\section{Cardinality}
\question{1}{$|{\N}|= |{\Z}|$ (10 pts)}
Define a bijection $f:\Z \to \N$.Explain why the function you define is a bijection.


\question{2} {Diagonal Argument (10 pts ***)}
This question prepares you to answer the next question.
Suppose there exist an onto function between the set $A=\{1,2,3,4\}$ and the powerset of $A$, i.e., $\mathcal{P}(A)$. For example, the following table represent a function $f:A\to \mathcal{P}(A)$

\begin{center}

\begin{tabular}{|l||*{5}{c|}}\hline
\diagbox{$n$}{$f(n)$}
&\makebox[3em]{1}&\makebox[3em]{2}&\makebox[3em]{3}
&\makebox[3em]{4}&\makebox[3em]{$f(n)$}\\\hline\hline
1&1&0&1&1& $\{1,3,4\}$\\\hline
2 &0&1&1&0&$\{2,3\}$\\\hline
3 &1&1&0&1&$\{1,2,4\}$\\\hline
4 &0&0&0&0&$\emptyset$\\\hline
\end{tabular}

\end{center}
Note that in the above table, we use $0$ to present an element NOT in a set and 1 for in. So we can see $1\mapsto \{1,3,4\}$ and the representation for the image is $1011$; and $4\mapsto \emptyset$ and the representation is $0000$.

Your task: Show the above function $f$ is not an onto function by Diagonal Argument. (Surely you will say this is too obvious since we are talking about a finite set, but don't forget this problem is for demonstration purpose.) You will need to construct a set $T$ that is in $\mathcal{P}(A)$ but not the the range of $f$. Let $d_1d_2d_3d_4$ be the 4-bit representation of $T$, the diagonal argument will tell us to do

\[
    d_n=\begin{cases}
     0, \quad\text{if $n\in f(n)$}.\\
     1, \quad\text{if $n\not\in f(n)$}
    \end{cases}
\]
Simply put it, you will just do the opposite on the diagonal. A conciser way to write it
\[
    T=\{n\mid n\not \in f(n) \}.
\]
 Or you will only include an element $n$ in $T$ if it is not in $f(n)$.


\textbf{Question: What is the resulting $T$ out of the above table/function $f$?}

\question{3}{Larger Cardinality Always Exists(10 pts ****)}
Show that $|A|< |\mathcal{P}(A)|$, for any set $A$, where $\mathcal{P}(A)$ denotes the powerset of $A$.

[Hint: Suppose an onto (or bijection) function f existed between $A$ and $\mathcal{P}(A)$ . Let \[
    T=\{a\in A|a\not\in f(a)\}
\]and show that no element $a$ can
exist for which $f(a)=T$. If that happens, will $T$ contains $a$?  ]

Fun fact: The argument goes somewhat like the \href{https://www.youtube.com/watch?v=nI-MMJiFpqo}{The Paradox of the Barber}(https://www.youtube.com/watch?v=nI-MMJiFpqo).

\question{4}{One to One Correspondence}
Determine whether each of these functions is a one-to-one correspondence (a.k.a. ``bijection'') from $\R$ to $\R$.
\begin{enumerate}
    \item $f(x)=2x+1$.
    \item $f(x)=x^2+1$.
    \item $f(x)=x^3+1$.
    \item $f(x)=\frac{x^2+1}{x^2+2}$.
    \item $f(x)=\frac{1}{x}+ \frac{1}{x-1}$.
\end{enumerate}

\question{5}{Reading and Writing: Schr\"oder-Berbstein Theorem}
\begin{enumerate}
    \item Define a notion of $|A|\leq |B|$ through functions.
    \item Define a notion of $|A|\geq |B|$ through functions.
    \item State Schr\"oder-Berbstein Theorem and give an application example. Why is Schr\"oder-Berbstein Theorem useful?
\end{enumerate}
\question{6}{Extra:Application of Schr\"oder-Berbstein Theorem (10 pts ***)}
Use the theorem to show that $(1,2)$ and $[1,2]$ has the same cardinality.
\end{document}
