%=====================ComS 511 LaTeX template, following the CMU 02-713 template================
%
% You don't need to use LaTeX or this template, but you must turn your homework in as
% a typeset PDF somehow.
%
% How to use:
%    1. Update your information in section "A" below
%    2. Write your answers in section "B" below. Precede answers for all
%       parts of a question with the command "\question{n}{desc}" where n is
%       the question number and "desc" is a short, one-line description of
%       the problem. There is no need to restate the problem.
%    3. If a question has multiple parts, precede the answer to part x with the
%       command "\part{x}".
%    4. If a problem asks you to design an algorithm, use the commands
%       \algorithm, \correctness, \runtime to precede your discussion of the
%       description of the algorithm, its correctness, and its running time, respectively.
%    5. You can include graphics by using the command \includegraphics{FILENAME}
%

\documentclass[11pt]{article}
\usepackage{amsmath,amssymb,amsthm}
\usepackage{times}
\usepackage{zi4}
\usepackage{graphicx}
\usepackage[margin=1in]{geometry}
\usepackage{fancyhdr}
\setlength{\parindent}{0pt}
\setlength{\parskip}{5pt plus 1pt}
\setlength{\headheight}{13.6pt}
\newcommand\question[2]{\vspace{.25in}\hrule\textbf{#1: #2}\vspace{.5em}\hrule\vspace{.10in}}
\renewcommand\part[1]{\vspace{.10in}\textbf{(#1)}}
\newcommand\algorithm{\vspace{.10in}\textbf{Algorithm: }}
\newcommand\correctness{\vspace{.10in}\textbf{Correctness: }}
\newcommand\runtime{\vspace{.10in}\textbf{Running time: }}
\newcommand\construction{\vspace{.10in}\textbf{Construction:}}

%%%%% theorem styles
\newtheorem{theorem}{Theorem}
\newtheorem{acknowledgement}[theorem]{Acknowledgement}
%\newtheorem{algorithm}[theorem]{Algorithm}
\newtheorem{axiom}{Axiom}
\newtheorem{case}[theorem]{Case}
\newtheorem{claim}[theorem]{Claim}
\newtheorem{conclusion}[theorem]{Conclusion}
\newtheorem{condition}[theorem]{Condition}
\newtheorem{conjecture}[theorem]{Conjecture}
\newtheorem{corollary}[theorem]{Corollary}
\newtheorem{criterion}[theorem]{Criterion}
\newtheorem{definition}[theorem]{Definition}
\newtheorem{example}[theorem]{Example}
\newtheorem{exercise}[theorem]{Exercise}
\newtheorem{lemma}[theorem]{Lemma}
\newtheorem{notation}[theorem]{Notation}
\newtheorem{problem}[theorem]{Problem}
\newtheorem{proposition}[theorem]{Proposition}
\newtheorem{remark}[theorem]{Remark}
\newtheorem{solution}[theorem]{Solution}
\newtheorem{summary}[theorem]{Summary}
%\newenvironment{proof}[1][Proof]{\textbf{#1.} }{\ \rule{0.5em}{0.5em}}

\newcommand{\Q}{\mathbb{Q}}
\newcommand{\R}{\mathbb{R}}
\newcommand{\C}{\mathbb{C}}
\newcommand{\Z}{\mathbb{Z}}
\newcommand{\N}{\mathbb{N}}

%% set builder helper
%
\usepackage{xparse}
\usepackage{mathtools}
\DeclarePairedDelimiterX{\set}[1]{\{}{\}}{\setargs{#1}}
\NewDocumentCommand{\setargs}{>{\SplitArgument{1}{;}}m}
{\setargsaux#1}
\NewDocumentCommand{\setargsaux}{mm}
{\IfNoValueTF{#2}{#1} {#1\nonscript\:\delimsize\vert\allowbreak\nonscript\:\mathopen{}#2}}%
\def\Set{\set*}%

\usepackage{diagbox} % for the table
\usepackage{hyperref}
%%% end of theorem styles
\pagestyle{fancyplain}
\lhead{\textbf{\NAME\ (\UISID)}}
\chead{\textbf{HW\HWNUM}}
\rhead{\textsc{CSC 302}, \today}
\begin{document}\raggedright
%Section A==============Change the values below to match your information==================
\newcommand\NAME{Xiang Huang}  % your name
\newcommand\UISID{\texttt{UISid}}     % your UIS id
\newcommand\HWNUM{3}              % the homework number
%Section B==============Put your answers to the questions below here=======================

% no need to restate the problem --- the graders know which problem is which,
% but replacing "The First Problem" with a short phrase will help you remember
% which problem this is when you read over your homeworks to study.

This document provide extra information about problem 3.

How to use this document:\textbf{You can read this document. However, when you write your own proof, you must set it aside.}

Other Materials that might help:
\begin{enumerate}
    \item A proof that uses interesting metaphor: ``humbel'' and ``cocky''.\href{http://mathcenter.oxford.emory.edu/site/math125/powerSetCardinality/}{Link.} (If you can not open the link directly. Try right click, then copy link.)
\end{enumerate}




\question{3}{Larger Cardinality Always Exists(10 pts ****)}
Show that $|A|< |\mathcal{P}(A)|$, for any set $A$, where $\mathcal{P}(A)$ denotes the powerset of $A$.

[Hint: Suppose an onto (or bijection) function f existed between $A$ and $\mathcal{P}(A)$ . Let \[
    T=\{a\in A|a\not\in f(a)\}
\]and show that no element $a$ can
exist for which $f(a)=T$. If that happens, will $T$ contains $a$?  ]

Fun fact: The argument goes somewhat like the \href{https://www.youtube.com/watch?v=nI-MMJiFpqo}{The Paradox of the Barber}(https://www.youtube.com/watch?v=nI-MMJiFpqo).


\begin{proof}
    Let $T$ be as mentioned above.
    Let`s assume the opposite and let $f(a)= T$. The we will have the following paradox:
     \begin{enumerate}
         \item $a$ can not be in $T$.  If $a\in T$, then $a\in f(a)$, then by $T$'s definition, $a\not \in T$.
         \item $a$ can not be not in $T$. If $a\not \in T$, then $a\in f(a)$, then by $T$'s definition, $a\in T$.
     \end{enumerate}
     Both cases results in contradiction.
     Therefore, the assumption $f(a)=T$ for some $a$ fails, which means $T$ does not have a preimage under $f$. Hence $f$ is not an onto function.
\end{proof}

\end{document}
