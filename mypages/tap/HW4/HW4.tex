%=====================ComS 511 LaTeX template, following the CMU 02-713 template================
%
% You don't need to use LaTeX or this template, but you must turn your homework in as
% a typeset PDF somehow.
%
% How to use:
%    1. Update your information in section "A" below
%    2. Write your answers in section "B" below. Precede answers for all
%       parts of a question with the command "\question{n}{desc}" where n is
%       the question number and "desc" is a short, one-line description of
%       the problem. There is no need to restate the problem.
%    3. If a question has multiple parts, precede the answer to part x with the
%       command "\part{x}".
%    4. If a problem asks you to design an algorithm, use the commands
%       \algorithm, \correctness, \runtime to precede your discussion of the
%       description of the algorithm, its correctness, and its running time, respectively.
%    5. You can include graphics by using the command \includegraphics{FILENAME}
%

\documentclass[11pt]{article}
\usepackage{amsmath,amssymb,amsthm}
\usepackage{times}
\usepackage{zi4}
\usepackage{graphicx}
\usepackage[margin=1in]{geometry}
\usepackage{fancyhdr}
\usepackage{tabularx}
\setlength{\parindent}{0pt}
\setlength{\parskip}{5pt plus 1pt}
\setlength{\headheight}{13.6pt}
\newcommand\question[2]{\vspace{.25in}\hrule\textbf{#1: #2}\vspace{.5em}\hrule\vspace{.10in}}
\renewcommand\part[1]{\vspace{.10in}\textbf{(#1)}}
\newcommand\algorithm{\vspace{.10in}\textbf{Algorithm: }}
\newcommand\correctness{\vspace{.10in}\textbf{Correctness: }}
\newcommand\runtime{\vspace{.10in}\textbf{Running time: }}
\newcommand\construction{\vspace{.10in}\textbf{Construction:}}
\newcommand\solution{\vspace{.10in}\textbf{Solution: }}

%%%%% theorem styles
\newtheorem{theorem}{Theorem}
\newtheorem{acknowledgement}[theorem]{Acknowledgement}
%\newtheorem{algorithm}[theorem]{Algorithm}
\newtheorem{axiom}{Axiom}
\newtheorem{case}[theorem]{Case}
\newtheorem{claim}[theorem]{Claim}
\newtheorem{conclusion}[theorem]{Conclusion}
\newtheorem{condition}[theorem]{Condition}
\newtheorem{conjecture}[theorem]{Conjecture}
\newtheorem{corollary}[theorem]{Corollary}
\newtheorem{criterion}[theorem]{Criterion}
\newtheorem{definition}[theorem]{Definition}
\newtheorem{example}[theorem]{Example}
\newtheorem{exercise}[theorem]{Exercise}
\newtheorem{lemma}[theorem]{Lemma}
\newtheorem{notation}[theorem]{Notation}
\newtheorem{problem}[theorem]{Problem}
\newtheorem{proposition}[theorem]{Proposition}
\newtheorem{remark}[theorem]{Remark}
%\newtheorem{solution}[theorem]{Solution}
\newtheorem{summary}[theorem]{Summary}
%\newenvironment{proof}[1][Proof]{\textbf{#1.} }{\ \rule{0.5em}{0.5em}}

\newcommand{\Q}{\mathbb{Q}}
\newcommand{\R}{\mathbb{R}}
\newcommand{\C}{\mathbb{C}}
\newcommand{\Z}{\mathbb{Z}}
\newcommand{\N}{\mathbb{N}}
%%% end of theorem styles
%\DeclareMathOperator{\gcd}{gcd}
%%%===========algorithm
%% algorithm
\usepackage{listings}
\usepackage{algorithm}
\usepackage[noend]{algpseudocode}
\usepackage{etoolbox}

\makeatletter
% start with some helper code
% This is the vertical rule that is inserted
\newcommand*{\algrule}[1][\algorithmicindent]{%
  \makebox[#1][l]{%
    \hspace*{.2em}% <------------- This is where the rule starts from
    \vrule height .75\baselineskip depth .25\baselineskip
  }
}

\newcount\ALG@printindent@tempcnta
\def\ALG@printindent{%
    \ifnum \theALG@nested>0% is there anything to print
    \ifx\ALG@text\ALG@x@notext% is this an end group without any text?
    % do nothing
    \else
    \unskip
    % draw a rule for each indent level
    \ALG@printindent@tempcnta=1
    \loop
    \algrule[\csname ALG@ind@\the\ALG@printindent@tempcnta\endcsname]%
    \advance \ALG@printindent@tempcnta 1
    \ifnum \ALG@printindent@tempcnta<\numexpr\theALG@nested+1\relax
    \repeat
    \fi
    \fi
}
% the following line injects our new indent handling code in place of the default spacing
\patchcmd{\ALG@doentity}{\noindent\hskip\ALG@tlm}{\ALG@printindent}{}{\errmessage{failed to patch}}
\patchcmd{\ALG@doentity}{\item[]\nointerlineskip}{}{}{} % no spurious vertical space
% end vertical rule patch for algorithmicx
\makeatother
%%% ==========end of algorithm
\pagestyle{fancyplain}
\lhead{\textbf{\NAME\ (\UISID)}}
\chead{\textbf{HW\HWNUM}}
\rhead{\textsc{CSC 302}, \today}
\begin{document}\raggedright
%Section A==============Change the values below to match your information==================
\newcommand\NAME{Xiang Huang}  % your name
\newcommand\UISID{\texttt{UISid}}     % your UIS id
\newcommand\HWNUM{4}              % the homework number
%Section B==============Put your answers to the questions below here=======================



% no need to restate the problem --- the graders know which problem is which,
% but replacing "The First Problem" with a short phrase will help you remember
% which problem this is when you read over your homeworks to study.



You should read Chapter 1-6 of \textit{A Friendly Introduction to Number Theory} to make sure you are familiar with the basic definitions before you start working on this assignment.

\question{1}{GCD, (10 pts)}
In the lecture I did not give a complete proof of the following fact:
\[
     Let n= qm +r, \ \text{where } 0\leq r\leq m-1, \ \text{then} \gcd(n,m)=\gcd(m,r)
\]
\question{2}{Extended Euclidean GCD Algorithm (10 pts)}
Please do some research online to find an implemention the Euclidean Algorithm. For each pair $(n,m)$ of the following numbers, apply your algorithm to calculate $\gcd(n,m)$.

You will also need to submit your source code on Canvas.

\begin{enumerate}
    \item[(a)] (35, 15).
    \item[(b)] (154878552455, 98871521).
    \item[(c)] (9784515182034984165, 3164547984351248).
    \item[(d)] (54321, 9876).
    \item[(e)] (12345, 67890).
\end{enumerate}

\question{3}{Linear Combination of Multiple(10 pts)}
Show that if $d|n$ and $d|m$, then for any $s,t\in \Z$,
\[
    d| (sn + tm).
\]

\question{4}{Divisibility and GCD, (Friendly) Exercise 7.1 and 7.2. (20 pts, 10 each,)}
Hint: use extended GCD algorithm to get a fancy way of writing one!
\begin{enumerate}
    \item[(a)] Suppose that $\gcd(a,b)=1$, and suppose further that $a$ divides the product $bc$. Show that $a$ must divide $c$.
    \item[(b)] Suppose that $\gcd(a,b)=1$, and suppose further that $a$ divides $c$ and $b$ divides $c$. Show that the product $ab$ must divide $c$.
\end{enumerate}

\end{document}
