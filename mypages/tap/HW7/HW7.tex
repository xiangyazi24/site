%=====================ComS 511 LaTeX template, following the CMU 02-713 template================
%
% You don't need to use LaTeX or this template, but you must turn your homework in as
% a typeset PDF somehow.
%
% How to use:
%    1. Update your information in section "A" below
%    2. Write your answers in section "B" below. Precede answers for all
%       parts of a question with the command "\question{n}{desc}" where n is
%       the question number and "desc" is a short, one-line description of
%       the problem. There is no need to restate the problem.
%    3. If a question has multiple parts, precede the answer to part x with the
%       command "\part{x}".
%    4. If a problem asks you to design an algorithm, use the commands
%       \algorithm, \correctness, \runtime to precede your discussion of the
%       description of the algorithm, its correctness, and its running time, respectively.
%    5. You can include graphics by using the command \includegraphics{FILENAME}
%

\documentclass[11pt]{article}
\usepackage{amsmath,amssymb,amsthm}
\usepackage{times}
\usepackage{zi4}
\usepackage{graphicx}
\usepackage[margin=1in]{geometry}
\usepackage{fancyhdr}
\usepackage{hyperref}
\setlength{\parindent}{0pt}
\setlength{\parskip}{5pt plus 1pt}
\setlength{\headheight}{13.6pt}
\newcommand\question[2]{\vspace{.25in}\hrule\textbf{#1: #2}\vspace{.5em}\hrule\vspace{.10in}}
\renewcommand\part[1]{\vspace{.10in}\textbf{(#1)}}
\newcommand\algorithm{\vspace{.10in}\textbf{Algorithm: }}
\newcommand\correctness{\vspace{.10in}\textbf{Correctness: }}
\newcommand\runtime{\vspace{.10in}\textbf{Running time: }}
\newcommand\construction{\vspace{.10in}\textbf{Construction:}}
\newcommand\solution{\vspace{.10in}\textbf{Solution: }}

%%%%% theorem styles
\newtheorem{theorem}{Theorem}
\newtheorem{acknowledgement}[theorem]{Acknowledgement}
%\newtheorem{algorithm}[theorem]{Algorithm}
\newtheorem{axiom}{Axiom}
\newtheorem{case}[theorem]{Case}
\newtheorem{claim}[theorem]{Claim}
\newtheorem{conclusion}[theorem]{Conclusion}
\newtheorem{condition}[theorem]{Condition}
\newtheorem{conjecture}[theorem]{Conjecture}
\newtheorem{corollary}[theorem]{Corollary}
\newtheorem{criterion}[theorem]{Criterion}
\newtheorem{definition}[theorem]{Definition}
\newtheorem{example}[theorem]{Example}
\newtheorem{exercise}[theorem]{Exercise}
\newtheorem{lemma}[theorem]{Lemma}
\newtheorem{notation}[theorem]{Notation}
\newtheorem{problem}[theorem]{Problem}
\newtheorem{proposition}[theorem]{Proposition}
\newtheorem{remark}[theorem]{Remark}
%\newtheorem{solution}[theorem]{Solution}
\newtheorem{summary}[theorem]{Summary}

%\newenvironment{proof}[1][Proof]{\textbf{#1.} }{\ \rule{0.5em}{0.5em}}

\newcommand{\Q}{\mathbb{Q}}
\newcommand{\R}{\mathbb{R}}
\newcommand{\Z}{\mathbb{Z}}
\newcommand{\N}{\mathbb{N}}
\DeclareMathOperator*{\dprime}{{\prime \prime}}
%%% end of theorem styles

%%%===========algorithm
%% algorithm
\usepackage{listings}
\usepackage{algorithm}
\usepackage[noend]{algpseudocode}
\usepackage{etoolbox}

\makeatletter
% start with some helper code
% This is the vertical rule that is inserted
\newcommand*{\algrule}[1][\algorithmicindent]{%
  \makebox[#1][l]{%
    \hspace*{.2em}% <------------- This is where the rule starts from
    \vrule height .75\baselineskip depth .25\baselineskip
  }
}

\newcount\ALG@printindent@tempcnta
\def\ALG@printindent{%
    \ifnum \theALG@nested>0% is there anything to print
    \ifx\ALG@text\ALG@x@notext% is this an end group without any text?
    % do nothing
    \else
    \unskip
    % draw a rule for each indent level
    \ALG@printindent@tempcnta=1
    \loop
    \algrule[\csname ALG@ind@\the\ALG@printindent@tempcnta\endcsname]%
    \advance \ALG@printindent@tempcnta 1
    \ifnum \ALG@printindent@tempcnta<\numexpr\theALG@nested+1\relax
    \repeat
    \fi
    \fi
}
% the following line injects our new indent handling code in place of the default spacing
\patchcmd{\ALG@doentity}{\noindent\hskip\ALG@tlm}{\ALG@printindent}{}{\errmessage{failed to patch}}
\patchcmd{\ALG@doentity}{\item[]\nointerlineskip}{}{}{} % no spurious vertical space
% end vertical rule patch for algorithmicx
\makeatother
%%% ==========end of algorithm
\pagestyle{fancyplain}
\lhead{\textbf{\NAME\ (\UISID)}}
\chead{\textbf{HW\HWNUM}}
\rhead{\textsc{CSC 302}, \today}
\begin{document}\raggedright
%Section A==============Change the values below to match your information==================
\newcommand\NAME{Xiang Huang}  % your name
\newcommand\UISID{\texttt{UISid}}     % your ISU id
\newcommand\HWNUM{7}              % the homework number
%Section B==============Put your answers to the questions below here=======================



% no need to restate the problem --- the graders know which problem is which,
% but replacing "The First Problem" with a short phrase will help you remember
% which problem this is when you read over your homeworks to study.
\question{1}{Combinatorial Proof via Bijection}
We discuss the idea of proving the following equation
\[\label{eq:odd_even}
    \sum_{k \ odd}\binom{n}{k} = \sum_{k\ even}\binom{n}{k}, \tag{1}
\]
in class. Roughly, let $[n]=\{1,2,\cdots, n\}$ be the universal set. We want to construct a bijection between the subsets of even sizes ($E$) and those of odd sizes ($O$). We define a function $f:E\to O$ as follows.
\[
    f(s)= \begin{cases}
         s\setminus \{1\}, \quad\text{if $1\in s$}.\\
         s\cup \{1\}, \quad\text{ohterwise.}
     \end{cases}
 \]
Argue that $f$ is a bijection and proof Equation (\ref{eq:odd_even}).

\question{2}{The selected and the unselected}
Prove the binomial identity
\[
    \binom{n}{k}=\binom{n}{n-k}.
\]
\question{3}{Half-and-half spilit}
Prove the binomial identity
\[
    \binom{n}{0}^2 + \binom{n}{1}^2 + \binom{n}{2}^2 + \cdots \binom{n}{n}^2 = \binom{2n}{n}.
\]
Hint: The right-hand side ask you to do $2n$ choose $n$. You split the $2n$ element into two group, each with $n$ element. To pick choose $n$ elements, you will need to pick $k$ element from one group and $n-k$ elements from another group, where $k\in \{0,1,\cdots, n\}$. You need to apply the result of question 2.

\question{4}{Generazation of Question 3}
Prove the identity
\[
    \sum_{k=0}^r \binom{n}{k}\binom{m}{r-k} = \binom{m+n}{r}.
\]

\question{5}{Generating Function}
Please deduct the ordinary generating function (OGF) of the following sequences.
\begin{enumerate}
    \item[(a)] $a_n = 2$, for $n\in \N$.
    \item[(b)] $a_n= n$, for $n\in \N$.
    \item[(c)] $a_n=3^n$, if $n \in \N$.
\end{enumerate}
\end{document}
